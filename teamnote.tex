\documentclass[10pt,landscape,a4paper,twocolumn]{article}

\setlength{\columnsep}{20pt}

\usepackage[left=1.2cm, right=1.2cm, top=1cm, bottom=1.3cm]{geometry}
\usepackage{amsmath}
\usepackage{amssymb}
\usepackage{fontspec}
\usepackage{kotex}
\usepackage{graphicx}
\usepackage{setspace}
\usepackage{listings}
\usepackage{comment}
\usepackage{import}
\usepackage{wrapfig}
\usepackage{url}
\usepackage{array}
\usepackage[normal]{engord}
\usepackage[svgnames,table]{xcolor}
\usepackage{courier}
\usepackage[T1]{fontenc}

\setmainhangulfont{NanumMyeongjo}

\definecolor{dkgrey}{RGB}{127, 127, 127}

\lstset{
    basicstyle=\footnotesize\ttfamily,
    breaklines=true,
    breakindent=1.1em,
    numbers=left,
    numberstyle=\footnotesize\ttfamily\color{dkgrey},
    numbersep=5pt
%   frame=trbl
}


\begin{document}


\section{Setting}
\subsection{vimrc}
\lstinputlisting{src/setting/vimrc}


\section{Math}

\subsection{basic arithmetic}
\lstinputlisting{src/math/basic-arithmetic.cpp}

\subsection{sieve methods : prime, divisor, euler phi}
\lstinputlisting{src/math/sieve.cpp}

\subsection{primality test}
\lstinputlisting{src/math/primality-test.cpp}

\subsection{chinese remainder theorem}
\lstinputlisting{src/math/chinese-remainder.cpp}

\subsection{burnside's lemma}
경우의 수를 세는데, 특정 transform operation(회전, 반사, ..)해서 같은 경우들은 하나로 친다.
전체 경우의 수는?

- 각 operation마다 이 operation을 했을 때 변하지 않는 경우의 수를 센다
(단, ``아무것도 하지 않는다''라는 operation도 있어야 함!)

- 전체 경우의 수를 더한 후, operation의 수로 나눈다. (답이 맞다면 항상 나누어 떨어져야 한다)

\section{Data Structure}

\subsection{fenwick tree}



\end{document}
